

\documentclass[12pt]{article}


\pagestyle{plain}

%packages
\usepackage{latexsym}
\usepackage{amsfonts}
\usepackage{amsmath}
\usepackage{amsthm}
\usepackage{amssymb}
\usepackage{fullpage}
\usepackage{changepage}
\usepackage{listings}
\usepackage{comment}

\usepackage[onehalfspacing]{setspace}

\setcounter{tocdepth}{1}

%environments
\newtheorem{thm}{Theorem}[section]
\newtheorem{lem}[thm]{Lemma}
\newtheorem{cor}[thm]{Corollary}
\newtheorem{prop}[thm]{Proposition}
\newtheorem{conj}[thm]{Conjecture}
\newtheorem{prob}[thm]{Problem}
\newtheorem{ques}[thm]{Question}
\newtheorem{exa}[thm]{Example}

%-- notation -- Check these out!!
\newcommand{\noqed}{\begin{flushright}$\Box$\end{flushright}}
\newcommand{\ba}{\[\begin{aligned}}
\newcommand{\ea}{\end{aligned}\]}
%numbers
\newcommand{\mc}[1]{\mathcal{#1}}
\newcommand{\N}{\mathbb{N}}
\newcommand{\Z}{\mathbb{Z}}
\newcommand{\R}{\mathbb{R}}
\newcommand{\C}{\mathbb{C}}
\newcommand{\Q}{\mathbb{Q}}
%sets
\newcommand{\setof}[1]{\left\{#1\right\}}
\newcommand{\setdef}[2]{\left\{#1\mid#2\right\}}
\newcommand{\floor}[1]{\lfloor#1\rfloor}
%matrix
\newcommand{\pmat}[1]{\begin{pmatrix}#1\end{pmatrix}}
%groups
\newcommand{\gen}[1]{\langle#1\rangle} %generator
\newcommand{\ol}[1]{\overline{#1}} %overline
\newcommand{\bo}[1]{\textbf{#1}}%bold
%topology
\newcommand{\Lk}{\text{Lk}}
\newcommand{\St}{\text{St}}
\newcommand{\dl}{\text{dl}}
\newcommand{\Bary}{\text{Bary }}
\newcommand{\skel}{\text{skel }}
\newcommand{\Baryn}[1]{\text{Bary}^{#1}\ }
\newcommand{\skeln}[1]{\text{skel}^{#1}\ }

%spacing
%\setlength{\parindent}{0in}
%\setlength{\parskip}{3mm}


\begin{document}

\title{Hamiltonian paths through the complete graph of the dihedral group}

\author{Kat Cannon-MacMartin}

%\date{September 8, 2014}

\maketitle

\begin{abstract}
  \noindent
  This paper investigates Hamiltonian paths through the
  complete graph of the dihedral group and their accompanying multisets
  of edge labels. Foundational work in this area includes
  Buratti, Horak, and Rosa's conjecture about Hamiltonian paths in graphs
  labeled with the integers, as well as Seamone and Stevens' generalization
  of this concept to arbitrary groups. Extending from these previous results,
  we provide a complete classification of realizable multisets containing two distinct elements,
  and supply partial results and a conjecture about realizable multisets containing
  three distinct elements.
\end{abstract}
\tableofcontents
\pagebreak

\section{Introduction}
This paper investigates Hamiltonian paths through the complete graph labeled
with the elements of the dihedral group, using an induced edge labeling
as described below. In particular, we endeavor to classify
which multisets of edges may arise from a valid Hamiltonian path, and which do not.

The dihedral group $D_{2m}$ represents the symmetries of a regular $m$-sided polygon.
The element $u$ represents a rotational symmetry, while $v$ represents a reflection,
defined as follows:
$$D_{2m} = \gen{u,\ v : u^m = e = v^2,\ vu = u^{-1}v}$$ 
Separating the elements of $D_{2m}$ by whether or not they contain a
reflection $v$, yields the subgroup
containing only rotations, $C_m = \setdef{u^n}{0 < n < m}$ and the coset formed by
adding a reflection to each rotation $C_mv$. It follows
that $C_m \cup C_mv = D_{2m}$ and $C_m \cap C_mv = \emptyset$.

Consider the the elements of an arbitrary finite group $G$ with order $n$ as vertex labels for the complete graph
$K_n$ with $n$ vertices. Edges are labeled with the difference between elements
such that an edge $\setof{x, y}$ is labeled $\setof{xy^{-1}, x^{-1}y}$. This is the induced
labeling employed throughout this paper.

In the specific case of the dihedral group, we adopt
a shorthand for representing edge labels using only a
single element as follows. Edges with differences $\setof{u^n, u^{-n}}$ are
labeled $\ol{u^n}$ with $n \leq \floor{\frac{m}{2}}$, while edges
with reflection differences $\setof{u^nv, u^nv}$ are labeled $\ol{u^nv}$.
Note that reflection elements are their own inverses.

\subsection{Background}
The basis for the Buratti-Horak-Rosa (BHR) conjecture was first proposed by Marco Buratti
in a private correspondence with Alex Rosa. Buratti's idea was generalized
by Rosa and Peter Horak, first published by them in \cite{BHR} as a conjecture about integers.
In this case, the problem considers a complete graph $K_v$ on the integers $\setof{0, 1, \dots, v - 1}$.
Edges are labeled such that edge $\setof{x, y}$ has label $\min(|x - y|,\ v - |x - y|)$.
In \cite{NewResult},
a paper investigating results related to Buratti's problem, Anita Pasotti and Marco Pellegrini
provide a succinct and clear statement of the BHR conjecture, which is paraphrased below.

\begin{conj}[M. Buratti, P. Horak, and A. Rosa]\label{BHR}
  Let $L$ be a multiset of $v-1$ positive integers
  not exceeding $\floor{\frac{v}{2}}$. Then there exists a Hamiltonian path $H$ of
  $K_v$ labeled with $L$ if, and only if, the following condition holds:
  \begin{equation}\label{BHRcondition}
    \left. \begin{array}{c}
      \textrm{for any divisor $d$ of $v$, the number of multiples of $d$} \\
      \textrm{appearing in $L$ does not exceed $v-d$.}
    \end{array}\right.
  \end{equation}
\end{conj}

\subsection{Results in cyclic groups}
The graph $K_v$ on integers $\setof{0, 1, \dots, v-1}$ described in the BHR conjecture is in
fact the complete graph of the cyclic group of order $v$, using the labeling for arbitrary groups
described above.
Consequently, in pursuit of the BHR conjecture, a range of results have been achieved in cyclic groups.
The statement is trivially true when $L$ contains exactly one distinct element, and in \cite{BHR} it is proven
to be true when $L$ contains exactly two distinct elements. A variety of cases have been proven when
$L$ contains exactly three distinct elements in \cite{123}, \cite{NewResult}, and \cite{OnBHR}.
Furthermore, Mariusz Meszka showed via computer program that the conjecture is true for all $v \leq 18$.
In \cite{NewMethods}, the most recent significant paper on the topic, the statement is proven for
a wide range of lists with arbitrary length.

\subsection{Generalization beyond the integers}
In \cite{SandS}, Ben Seamone and Brett Stevens generalize the BHR
conjecture beyond the integers, expanding to
a theorem about spanning trees in any finite group. For a complete graph $K$ with
vertices labeled by the elements of finite group $G$, Seamon and Stevens define a
multiset of induced edge labels $L$ as \textit{realizable} if there exists a spanning
tree in $K$ with exactly the edge labels in $L$.

\begin{thm}[B. Seamone and B. Stevens]\label{sst}
  Let $G$ be a finite group of order $n$. A label set $L$ of size $n - 1$
  is realizable as a $G$-spanning tree if and only if,
  \begin{equation}\label{sstcondition}
    \left. \begin{array}{c}
      \textrm{for any $H < G$, there are at most $n - [G:H]$} \\
      \textrm{elements in $L$ which lie in $H$.}
    \end{array}\right.
  \end{equation}
\end{thm}

With this generalization in mind, we endeavor to explore the implications of Theorem \ref{sst}
as it applies to the dihedral group. Namely, which multisets of dihedral group elements can be
realized as Hamiltonian paths, and which cannot? For our purposes, we define a multiset of edge labels
as \textit{realizable} if there exists a Hamiltonian path using these labels, a stricter condition
than that of Seamone and Stevens.
Before investigating this issue, we address further notation
for expressing multisets of edge labels in the graph of the dihedral group.

\subsection{Multiset notation}
In this paper, the ``label sets'' mentioned in Theorem \ref{sst} will be represented
as multisets of edge-labels in the graph $K_{2m}$ as described in the introduction.
When addressing a multiset $M$ of edge labels, it is not
possible to use traditional exponent-based multiset notation to denote the quantity of
specific elements as the dihedral group already differentiates its elements by exponent.
In its place, we use ``$a \cdot \ol{x}$'' to denote that a multiset
contains $a$ copies of the edge-label $\ol{x}$.

\begin{exa}
  The multiset of edge labels for $D_8$ containing $2$ copies of the label $\ol{u}$, $3$ copies of $\ol{u^2v}$,
  and $2$ copies of $\ol{v}$ is denoted:
  $$L = \setof{2 \cdot \ol{u},\ 2 \cdot \ol{v},\ 3 \cdot \ol{u^2v}}$$
\end{exa}

The following sections prove general statements about this problem in the dihedral group,
as direct extensions of Theorem \ref{sst}. Additionally, we investigate realizable and
not realizable multisets in detail, leading to a complete classification of multisets
with two distinct elements, and a partial classification of multisets with three
distinct elements.

\section{General statements on the dihedral group}
The application of Theorem \ref{sst} to the dihedral group produces a number
of intuitive results without any extra work.
To get a taste of these direct applications, two corollaries
of Theorem \ref{sst} follow, with supplementary proofs relating specifically to
the dihedral group.

\begin{cor} \label{leqm} In the dihedral group $D_{2m}$, multisets $M$ of the form
\begin{itemize}
\item $M = \setof{a \cdot u^nv, \dots}$ for any $n < m$
\item $M = \setof{a \cdot u^{m/2}, \dots}$ if $m$ is even
\end{itemize}
are not realizable for $a > m$.
\end{cor}
\begin{proof}
  In the dihedral group $D_{2m}$ the only elements with a difference of
  exactly $u^nv$ are pairs $\setof{u^i, u^{i + n}v}$ for any $i < m$. There are exactly $m$
  such pairs of elements in $D_{2m}$, therefore there are exactly $m$ edges
  labeled $\ol{u^nv}$ in the graph $K_{2m}$ for any $n < m$. It follows
  that a realizable multiset may not contain more than $m \cdot \ol{u^nv}$.

  Similarly, when $m$ is even, there are exactly $m$ pairs with a difference $\ol{u^{m/2}}$.
  Two elements with a difference of $\ol{u^n}$ for any $n$ must both be members of either the
  subgroup $C_m$ or the coset $C_mv$. A difference of $\ol{u^{m/2}}$ with $m$ even
  is an involution, therefore the number of edges in $K_{2m}$ labeled $\ol{u^{m/2}}$ is exactly
  $\frac{|C_m|}{2} + \frac{|C_mv|}{2} = m$. It follows that a realizable
  multiset may not contain more than $m \cdot \ol{u^{m/2}}$ when $m$ is even.
\end{proof}

\begin{cor} \label{needv} In the dihedral group $D_{2m}$, multisets must contain
  at least one element $\ol{u^nv}$ (for any $n < m$) to be realizable.
\end{cor}
\begin{proof}
  In the graph $G_{2m}$, any edge labeled $\ol{u^n}$ (for any $n < m$) connects two elements of
  the subgroup $C_m$ \textit{or} two elements of the coset $C_mv$. Any realizable multiset
  must contain at least one edge connecting an element of $C_m$ to an element of $C_mv$,
  and all such edges are labeled $\ol{u^nv}$ (for any $n < m$).
\end{proof}

The theorem of Seamone and Stevens is a good starting point, but obviously not completely sufficient.
As any Hamiltonian path is a spanning tree, any realizable difference set must meet
the condition of Theorem \ref{sst}, however not all qualifying
difference sets will suffice for our purposes.

\begin{exa}\label{bsst}
  In the dihedral group $D_8$ the multiset $L = \setof{4 \cdot \ol{v},\ \ol{uv},\ 2 \cdot \ol{u^3v}}$
  meets the necessary condition of Theorem \ref{sst} but is not realizable as a Hamiltonian
  path through $K_{2m}$.
\begin{proof}
  Differences of $\ol{v}$ may not appear adjacent in the sequence of differences, as multiplication
  by $v$ is an involution. Therefore, every odd-indexed difference in the sequence must be $\ol{v}$.
  It follows that the sequence of differences $\ol{v}, \ol{u^3v}, \ol{v}, \ol{uv}$ must be present
  in the realization of $L$. These differences, when combined, yield the identity, indicating
  that the first element in a sequence that realizes these differences will be the same as the
  last element in the sequence. This indicates a repeated element in the realization of $L$, therefore
  the multiset $L$ is not realizable.
\end{proof}
\end{exa}

A multiset which meets the condition proposed by Seamone and Stevens in Theorem \ref{sst}
will henceforth be referred to as an \textit{admissible} multiset, which differentiates it from a
\textit{realizable} multiset that can definitely form a Hamiltonian path.
All realizable multisets are admissible, but as shown above, not all admissible sets are realizable.

\section{Theorems}

It is established by Example \ref{bsst} that multisets of edge labelings must be held
to a higher standard than that of Theorem \ref{sst} to fulfill our criteria.
The following sections attack two and three element \textit{admissible} multisets
in the dihedral group in an attempt to classify which are realizable and which are not.

The easiest method for proving that a multiset is realizable
is providing a sequence (or generalized sequence) that realizes the desired labels.
Throughout the following proofs, the specialized notation $\bo{s}_i, \bo{t}_i$ is used
when expressing some sequences. This notation leverages the proof of Corollary \ref{leqm},
which establishes that any label $\ol{u^nv}$ is applied only to edges between
elements $u^i, u^{i+n}v$ for any $i$. Consequently, the notational conventions
$\bo{s}_i, \bo{t}_i$ correspond to the two possible orderings of such sequences,
with $\bo{s}_i = u^i, u^{i+n}v$ and $\bo{t}_i = u^{i+n}v, u^i$ for a given $n$. It follows
from the proof of Corollary \ref{leqm} that any multiset containing $a$ elements $\ol{u^nv}$ must contain
exactly $a$ of such pairings.

\subsection{Two-element constructions}\label{sectwo}

Two-element multisets are generally easy to realize. Most significantly, as all multisets for a given group
must contain the same number of labels, any restrictions on the quantity of one element necessarily restricts
both. Furthermore, as shown in Example \ref{bsst},
the same reflection element $\ol{u^nv}$ may not appear sequentially in a sequence of differences.
This means that any repeated reflection elements must be at least alternated with another element in the set.
In two-element multisets, there is only one other element with which to alternate, greatly restricting possible orderings.
By Corollary \ref{needv}, realizable multisets must contain at least one reflection element,
necessitating that this restriction is present in any two-element realization.

This way in which each element of a two-element multiset restricts the other results in easy extensions of possibility
proofs (realizations) into equivalent impossibility proofs (if and only if). Below is a demonstrative proof on perhaps
the simplest multiset form in the dihedral group.

\begin{lem} \label{u+v} In the dihedral group $D_{2m}$, multisets of the form
$$M = \setof{a \cdot \ol{u},\ b \cdot \ol{v}}$$
are realizable for exactly the values
$$\setdef{a, b}{ m - 1 \leq a \leq 2m - 2,\ b = (2m - 1) - a}.$$
\end{lem}

\begin{proof}
  $(\Rightarrow)$
  For $b = 1, 2$ multisets of the given form are
  realized by the sequences
  \ba
  S_1 &= e, u, \dots , u^{m-1}, u^{m-1}v, u^{m-2}v, \dots, v\\
  \text{Labels} &: \ol{u}, \ol{u}, \dots, \ol{v}, \ol{u}, \ol{u}, \dots, \ol{u}\\
  S_2 &= e, v, uv, \dots, u^{m-1}v, u^{m-1}, u^{m-2},\dots, u\\
  \text{Labels} &: \ol{v}, \ol{u}, \ol{u}, \dots, \ol{v}, \ol{u}, \ol{u}, \dots, \ol{u}
  \ea

  Define the strings $\bo{s}_i, \bo{t}_i$ to be $\bo{s}_i = u^i, u^iv$ and
  $\bo{t}_i = u^iv, u^i$ for any $i < m$. For $2 < b \leq m$, multisets
  of the given form are realized by the sequences
  $$
  S = \begin{cases}
    \bo{s}_0, \bo{t}_1, \bo{s}_2, \bo{t}_3, \dots, \bo{s}_{b-2}, u^{b-1}v, \dots, u^{m-1}v, u^{m-1}, u^{m-2}, \dots, u^{m-b+2} & \text{b is even}\\
    \bo{s}_0, \bo{t}_1, \bo{s}_2, \bo{t}_3, \dots, \bo{t}_{b-2}, u^{b-1}, \dots, u^{m-1}, u^{m-1}v, u^{m-2}v, \dots, u^{m-b+2}v & \text{b is odd}
  \end{cases}
  $$
  these sequences both yield a similarly formed sequence of labels
  $$ \ol{v}, \ol{u}, \ol{v}, \ol{u}, \dots, \ol{v}, \ol{u}, \ol{u}, \dots, \ol{u}, \ol{v}, \ol{u}, \dots, \ol{u}.$$

  $(\Leftarrow)$
  Multisets of the given form are not realizable for $a < m - 1$, as Corollary \ref{leqm}
  dictates that $b \leq m$. Multisets of the given form are not realizable for $a > 2m - 2$,
  as Corollary \ref{needv} dictates that $b \geq 1$.
\end{proof}

Generalized forms of sequences can be difficult to digest. An example illustrating the generalized sequences of Lemma
\ref{u+v} follows.

\begin{exa}
  In the dihedral group $D_{10}$, the multiset
  $$M = \setof{5 \cdot \ol{u},\ 4 \cdot \ol{v}}$$
  is realized by the sequence
  \ba
  &Sequence : e, v, uv, u, u^2, u^2v, u^3v, u^4v, u^4, u^3\\
  &Labels : \ol{v}, \ol{u}, \ol{v}, \ol{u}, \ol{v}, \ol{u}, \ol{u}, \ol{v}, \ol{u}
  \ea
\end{exa}

Lemma \ref{u+v} provides an excellent classification of the two simplest edge labelings
in the dihedral group. This classification can be extended greatly, however, using the
automorphism properties of the group. A corollary of Lemma \ref{u+v} follows,
expanding its implications.

\begin{cor} \label{un} In the dihedral group $D_{2m}$, multisets of the form
$$M = \setdef{a \cdot \ol{u^{n_1}},\ b \cdot \ol{u^{n_2}v}}{n_1 \nmid m,\ n_1, n_2 < m}$$
are realizable for exactly the values
$$\setdef{a, b}{ m - 1 \leq a \leq 2m - 2,\ b = (2m - 1) - a}.$$
\end{cor}
\begin{proof}
  Consider the elements $u^{n_1}, u^{n_2}v$ as generators of the group $D_{2m}$.
  Because $n_1 \nmid m$, $u^{n_1}$ generates the subgroup $C_m$, and it follows
  that $\gen{u^{n_1}, u^{n_2}v}$ generates $D_{2m}$. Following the rules set
  in the presentation of $D_{2m}$ we have
  \ba
  (u^{n_1})^m = u^{n_1m} &= e\\
  (u^{n_2}v)^2 = u^{n_2}vu^{n_2}v = v^2 &= e
  \ea
  and for the second rule:
  $$u^{n_2}vu^{n_1} = u^{n_2 - n_1}v$$
  We can present the group
  $$D_{2m} = \gen{u^{n_1},\ u^{n_2}v : (u^{n_1})^m = e = (u^{n_2}v)^2,\ u^{n_2}vu^{n_1} = u^{n_2-n_1}v}$$
  therefore there exists an automorphism $\phi$ such that $\phi : u^{n_1} \to u$ and $\phi : u^{n_2}v \to v$.

  Applying an automorphism $\phi$ to a sequence $S$ yields a new sequence $S_{\phi}$ with edge labels
  equivalent to applying $\phi$ to the edge labels of $S$. Therefore the sequences presented
  in Lemma \ref{u+v} imply the existence of sequences satisfying multisets of the form $M$, for
  the same values $a,b$.
\end{proof}

Given the ease of applying automorphism to realizations of two-element multisets, the natural path
forward is to classify simple two-element constructions that can then be readily expanded
using automorphisms. If Lemma \ref{u+v} provides one possible form of two-element multiset,
a rotation element and a reflection element, then we are left with only one other possible form:
two reflection elements. Recall that a two-element multiset containing two different rotation elements
is not realizable per Corollary \ref{needv}.

\begin{lem} \label{unv} In the dihedral group $D_{2m}$, multisets of the form
$$M = \setof{a \cdot \ol{v},\ b \cdot \ol{uv}}$$
are realizable for exactly the values
$$\setdef{a,\ b}{a \in \setof{m, m-1},\ b = 2m - (a + 1)}.$$
\end{lem}

\begin{proof}
  $(\Rightarrow)$
  At $a = m - 1, b = m$ multisets of the given form are realized by the sequence
 
  \ba
  S_1 &= e, uv, u, u^2v, u^2, \dots , u^{m-1}v, u^{m-1}, v\\
  \text{Labels} &: \ol{uv}, \ol{v}, \ol{uv}, \ol{v}, \dots, \ol{v}, \ol{uv}
  \ea
 
  At $a = m, b = m - 1$ multisets of the given form are realized by reversing
  the entire sequence $S_1$ with the exception of the initial identity element, as below
  \ba
  S_2 &= e, v, u^{m-1}, u^{m-1}v, u^{m-2}, u^{m-2}v, \dots , u, uv\\
  \text{Labels} &: \ol{v}, \ol{uv}, \ol{v}, \ol{uv}, \dots, \ol{uv}, \ol{v}
  \ea

  $(\Leftarrow)$
  Corollary \ref{leqm} dictates that $a,b \leq m$, therefore multisets with $a > m$ or
  $a < m - 1$ are not realizable.
\end{proof}

\begin{exa}
  In the dihedral group $D_{8}$, the multiset
  $$M = \setof{4 \cdot \ol{v},\ 3 \cdot \ol{uv}}$$
  is realized by the sequence
  \ba
  &Sequence : e, v, u^3, u^3v, u^2, u^2v, u, uv\\
  &Labels : \ol{v}, \ol{uv}, \ol{v}, \ol{uv}, \ol{v}, \ol{uv}, \ol{v}
  \ea
\end{exa}

As with Lemma \ref{u+v}, Lemma \ref{unv} can be generalized significantly through automorphisms.

\begin{cor} \label{uvv}
In the dihedral group $D_{2m}$, multisets of the form
$$M = \setdef{a \cdot \ol{u^{n_1}v},\ b \cdot \ol{u^{n_2}v}}{n_1 - n_2 \nmid m,\ n_1,n_2 < m}$$
are realizable for exactly the values
$$\setdef{a,\ b}{a \in \setof{m, m-1},\ b = 2m - (a + 1)}.$$
\end{cor}

\begin{proof}
  In the dihedral group, if a subset $\gen{x, y}$ generates the group, then there
  exists an automorphism $\phi$ that maps $\phi : x \to x'$, $\phi : y \to y'$
  for any $x', y'$ such that $|x| = |x'|,\ |y| = |y'|$ and $\gen{x', y'}$ generates
  the group. Therefore, for fixed $a,b$ there exists a map from the multiset presented in Lemma \ref{unv}
  to any multiset of the given form $M$, and by extension any multiset of the form $M$ is realizable for
  exactly the values of $a, b$ presented in Lemma \ref{unv}.
\end{proof}

Admissibility of multisets, as defined by Theorem \ref{sst}, is determined relatively easily
for two-element constructions. In order to form a spanning tree through a graph, every
element of the group must be reachable from the identity, traveling only along edges labeled
with the elements in the given multiset. This is equivalent to the elements in a multiset
generating the group. Using this property that a multiset must generate the group to be
admissible, we can easily reach the following theorem based on the results of
Lemmas \ref{u+v} and \ref{unv}.

\begin{thm} \label{two} In the dihedral group $D_{2m}$ all admissible multisets
  containing two distinct elements are realizable.
\end{thm}

\begin{proof}
  It is trivially true that a difference set must generate the group to be admissible.
  To generate $D_{2m}$, a two-element set must generate an element $u^{n_1}$
  such that $n_1 \nmid m$, as well as contain some element $u^{n_2}v$ for any $n_2 < m$.
  The element $u^{n_1}$ generates $C_m$, and when combined with the element $u^{n_2}v$
  generates $C_mv$.

  By this rule, multisets of the form
  $$M = \setof{a \cdot \ol{u^{n_1}},\ b \cdot \ol{u^{n_2}v}}$$
  are admissible only if $n_1 \nmid m$, exactly the range of values covered by
  Corollary \ref{un}.

  Multisets of the form
  $$M = \setof{a \cdot \ol{u^{n_1}v},\ b \cdot \ol{u^{n_2}v}}$$
  are admissible only if $n_1 - n_2 \nmid m$, as the element $u^n$ with $n \nmid m$ must be
  generated by multiplying $u^{n_1}vu^{n_2}v = u^{n_1 - n_2}$. This is exactly the range of
  values covered by Corollary \ref{uvv}.

  Lemmas \ref{u+v} and \ref{unv} cover all admissible values of $a, b$, therefore
  all admissible two-element multisets are realizable.
\end{proof}

As established previously, a Hamiltonian path is an example of a spanning tree,
and therefore a multiset must be admissible in order to be realizable. Consequently,
Theorem \ref{two} classifies all two-element multisets that are realizable as Hamiltonian
paths through the graph of the dihedral group.

\subsection{Three-element constructions}\label{secthree}
Three-element multisets pose a significantly greater challenge than two. While all admissible
two-element multisets proved to be realizable, Example \ref{bsst} already shows that the same
cannot be said for three-element constructions. One easy way to produce results is emulating
a two-element multiset by fixing the quantity of one element at a value, often $m$, and solving
for possible quantities of the other elements. This is easiest when the element fixed to
$m$ copies is a reflection. As they may not appear sequential, every other edge starting
with the first will be labeled with the fixed element. The following lemma and example leverage
this property for easy results.

\begin{lem} \label{fixc} In the dihedral group $D_{2m}$, multisets of the form
$$M = \setof{a \cdot \ol{u},\ b \cdot \ol{v},\ c \cdot \ol{uv}}$$
are realizable for the values
$$\setdef{a, b, c}{c = m,\ a + b + c = 2m - 1}$$
\end{lem}
\begin{proof}
  When $a = 0$ or $b = 0$, $M$ is realizable by Theorem \ref{two}.
  With strings $\bo{s}_i, \bo{t}_i$ such that $\bo{s}_i = u^i, u^{i+1}v$ and
  $\bo{t}_i = u^{i+1}v, u^i$, multisets of the given form are realized by the sequence
  $$
  S = \begin{cases}
    \bo{s}_0, \bo{s}_1, \dots, \bo{s}_b, \bo{t}_{b+1}, \bo{s}_{b+2}, \dots, \bo{s}_{m-2}, \bo{t}_{m-1} & \text{a is odd}\\
    \bo{s}_0, \bo{s}_1, \dots, \bo{s}_b, \bo{t}_{b+1}, \bo{s}_{b+2}, \dots, \bo{t}_{m-2}, \bo{s}_{m-1} & \text{a is even}\\
    \end{cases}
  $$
  The initial sequence of strings $\bo{s}_i$ are each labeled with with $\ol{uv}$, while the edge between each
  is labeled with $\ol{v}$. Strings $\bo{t}_i$ are similarly labeled with $\ol{uv}$, however alternating between
  $\bo{s}_i, \bo{t}_i$ yields labels of $\ol{u}$ between each. Therefore these sequences yield the labels:
  $$\ol{uv}, \ol{v}, \ol{uv}, \dots, \ol{v}, \ol{uv}, \ol{u}, \ol{uv}, \dots, \ol{u}, \ol{uv}$$
\end{proof}
\begin{exa}
  In the dihedral group $D_{10}$, the multiset
  $$M = \setof{2 \cdot \ol{u},\ 2 \cdot \ol{v},\ 5 \cdot \ol{uv}}$$
  is realized by the sequence
  \ba
  &Sequence : e, uv, u, u^2v, u^2, u^3v, u^4v, u^3, u^4, v\\
  &Labels : \ol{uv}, \ol{v}, \ol{uv}, \ol{v}, \ol{uv}, \ol{u}, \ol{uv}, \ol{u}, \ol{uv}
  \ea
\end{exa}

While Lemma \ref{fixc} uses a fixed number of labels $\ol{uv}$, we can produce similar results
from the same multiset form by simply fixing the other reflection label $\ol{v}$, as below.

\begin{lem} \label{fixb} In the dihedral group $D_{2m}$, multisets of the form
$$M = \setof{a \cdot \ol{u},\ b \cdot \ol{v},\ c \cdot \ol{uv}}$$
are realizable for the values
$$\setdef{a, b, c}{b = m,\ a + b + c = 2m - 1}$$
\end{lem}
\begin{proof}
  When $a = 0$ or $c=0$, $M$ is realizable by Theorem \ref{two}.
  With strings $\bo{s}_i, \bo{t}_i$ such that $\bo{s}_i = u^i, u^iv$ and
  $\bo{t}_i = u^iv, u^i$, multisets of the given form are realized by the sequence
  $$
  S = \begin{cases}
    \bo{s}_0, \bo{s}_{m-1}, \bo{s}_{m-2}, \dots, \bo{s}_{m-c}, \bo{t}_{m-c-1}, \bo{s}_{m-c-2}, \dots, \bo{s}_2, \bo{t}_1 & \text{a is odd}\\
    \bo{s}_0, \bo{s}_{m-1}, \bo{s}_{m-2}, \dots, \bo{s}_{m-c}, \bo{t}_{m-c-1}, \bo{s}_{m-c-2}, \dots, \bo{t}_2, \bo{s}_1 & \text{a is even}\\
  \end{cases}
  $$
  which produces the sequence of labels
  $$\ol{v}, \ol{uv}, \ol{v}, \dots, \ol{uv}, \ol{v}, \ol{u}, \ol{v}, \dots, \ol{u}, \ol{v}$$
\end{proof}
\begin{exa}
  In the dihedral group $D_{8}$, the multiset
  $$M = \setof{1 \cdot \ol{u},\ 4 \cdot \ol{v},\ 2 \cdot \ol{uv}}$$
  is realized by the sequence
  \ba
  &Sequence : e, v, u^3, u^3v, u^2, u^2v, uv, u\\
  &Labels : \ol{v}, \ol{uv}, \ol{v}, \ol{uv}, \ol{v}, \ol{u}, \ol{v}
  \ea
\end{exa}

In the interest of following the pattern laid out by the previous two theorems, the natural
next step is to fix the number of $\ol{u}$ labels to $m$. Unfortunately, this action
is not nearly as restrictive as fixing the quantity of reflection labels. Crucially,
rotation labels such as $\ol{u}$, may be repeated sequentially in a sequence of labels.
As a result, enforcing a specific number of $\ol{u}$ labels does not impose the same
structure on a sequence as fixing a reflection label. The $m$ $\ol{u}$ labels may appear
anywhere in the sequence, just as if there were $m-1$ or $m+1$ of them.

The absence of this structure poses a secondary challenge as well. When every other label,
starting with the first, is a reflection element, then any realizing sequence will necessarily
travel from the subgroup $C_m$ to the coset $C_mv$ and back every two elements. While this
traversal may happen more often as a result of additional reflection elements, the guaranteed
alternation severely limits which multisets can possibly be realized, making the work
that much easier.

Despite these limiting factors, some results can still easily be achieved. The following
theorem emulates previous proofs by loosely requiring that every other label is $\ol{u}$.
The specific form does not set $\ol{u}$ as the first label, however, so two consecutive
$\ol{u}$ labels appear.

\begin{lem} \label{fixaceven} In the dihedral group $D_{2m}$, multisets of the form
$$M = \setof{a \cdot \ol{u},\ b \cdot \ol{v},\ c \cdot \ol{uv}}$$
are realizable for the values
$$\setdef{a, b, c}{a = m,\ a + b + c = 2m - 1}$$
when $m$ and $c$ are even.
\end{lem}
\begin{proof}
  When $b = 0$ or $c = 0$, $M$ is realizable by Theorem \ref{two}.
  With strings $\bo{s}_i, \bo{t}_i$ such that $\bo{s}_i = u^i, u^{i+1}v$ and
  $\bo{t}_i = u^{i+1}v, u^i$, multisets of the given form are realized by the sequence
  $$S = \bo{s}_0, \bo{t}_1, \bo{s}_2, \dots, \bo{t}_{c-1}, u^c, u^{c+1}, u^{c+1}v, u^{c+2}v, \dots, u^{m-1}, u^{m-1}v, v$$
  The sequence splits the required multiset in two, covering all labels $\ol{uv}$ in the first half and labels $\ol{v}$
  in the second. Every other label is $\ol{u}$ in an alternating fashion, with two in sequence in the middle.
  This produces the following labels:
  $$\ol{uv}, \ol{u}, \ol{uv}, \dots, \ol{uv}, \ol{u}, \ol{u}, \ol{v}, \ol{u}, \dots, \ol{u}, \ol{v}, \ol{u}$$
\end{proof}
\begin{exa}
  In the dihedral group $D_{8}$, the multiset
  $$M = \setof{4 \cdot \ol{u},\ 1 \cdot \ol{v},\ 2 \cdot \ol{uv}}$$
  is realized by the sequence
  \ba
  &Sequence : e, uv, u^2v, u, u^2, u^3, u^3v, v\\
  &Labels : \ol{uv}, \ol{u}, \ol{uv}, \ol{u}, \ol{u}, \ol{v}, \ol{u}
  \ea
\end{exa}

The sequence provided in Lemma \ref{fixaceven} endeavors to alternate $\ol{u}$ labels by
separating $\ol{uv}$ and $\ol{v}$ labels. Taking the opposite approach yields results as well,
however. The following Lemmas produce realizations with sequences that group all reflection
labels except one at the beginning of the sequence. At the end of this section, differences
of only $\ol{u}$ are used to traverse what remains of either $C_m$ or $C_mv$. When all
remaining elements of the given subset have been exhausted, the final reflection
label is used to switch to the other subset, and the final elements are again
traversed using labels $\ol{u}$.

While these sequences are elegant in their form, the grouping of all reflection
elements limits them to multisets with similar quantities of labels $\ol{v}$ and $\ol{uv}$.

\begin{lem}\label{moreuv} In the dihedral group $D_{2m}$, multisets of the form
$$M = \setof{a \cdot \ol{u},\ b \cdot \ol{v},\ c \cdot \ol{uv}}$$
are realizable for the values
$$\setdef{a, b, c}{1 \leq c \leq m,\ c - b = 1,\ a + b + c = 2m - 1}$$
\end{lem}
\begin{proof}
  Multisets of the given form are realized by the sequence
  $$S = e, uv, u, u^2v, u^2, \dots, u^{c-1}v, u^{c-1}, u^{c}, \dots, u^{m-1}, v, u^{m-1}v, u^{m-2}v, \dots, u^{c}v$$
  which is labeled
  $$\ol{uv}, \ol{v}, \ol{uv}, \dots, \ol{v}, \ol{u}, \ol{u}, \dots, \ol{uv}, \ol{u}, \ol{u}, \dots, \ol{u}$$
\end{proof}
\begin{exa}
  In the dihedral group $D_{10}$, the multiset
  $$M = \setof{2 \cdot \ol{u},\ 3 \cdot \ol{v},\ 4 \cdot \ol{uv}}$$
  is realized by the sequence
  \ba
  &Sequence : e, uv, u, u^2v, u^2, u^3v, u^3, u^4, v, u^{4}v\\
  &Labels : \ol{uv}, \ol{v}, \ol{uv}, \ol{v}, \ol{uv}, \ol{v}, \ol{u}, \ol{uv}, \ol{u}
  \ea
\end{exa}

\begin{lem} \label{morev} In the dihedral group $D_{2m}$, multisets of the form
$$M = \setof{a \cdot \ol{u},\ b \cdot \ol{v},\ c \cdot \ol{uv}}$$
  are realizable for the values
  $$\setdef{a, b, c}{1 \leq b \leq m,\ b - c \in \setof{1, 2},\ a + b + c = 2m - 1}$$
\end{lem}
\begin{proof}
  Multisets of the given form are realized by the sequence
  $$
  S = \begin{cases}
    e, v, u^{m-1}, u^{m-1}v, u^{m-2}, u^{m-2}v, \dots, u^{m-b+1}, u^{m-b}, \dots, u, uv, u^2v, \dots, u^{m-b+1}v \\
    \text{in the case } b - c = 1\\
    e, v, u^{m-1}, u^{m-1}v, u^{m-2}, u^{m-2}v, \dots, u^{m-b+2}, u^{m-b+2}v, u^{m-b+1}v, \dots, uv, u, u^2, \dots, u^{m-b} \\
    \text{in the case } b - c = 2\\
  \end{cases}
  $$

  In the case $b - c = 1$, the sequence is labeled
  $$\ol{v}, \ol{uv}, \ol{v}, \dots, \ol{uv}, \ol{u}, \ol{u}, \dots, \ol{u}, \ol{v}, \ol{u}, \dots, \ol{u}$$
  In the case $b - c = 2$, the sequence is labeled
  $$\ol{v}, \ol{uv}, \ol{v}, \dots, \ol{uv}, \ol{v}, \ol{u}, \dots, \ol{u}, \ol{v}, \ol{u}, \dots, \ol{u}$$
\end{proof}

\begin{exa}
  In the dihedral group $D_{10}$, the multiset
  $$M = \setof{ 3 \cdot \ol{u},\ 4 \cdot \ol{v},\ 2 \cdot \ol{uv}}$$
  is realized by the sequence
  \ba
  &Sequence : e, v, u^4, u^4v, u^3, u^3v, u^2v, uv, u, u^2\\
  &Labels : \ol{v}, \ol{uv}, \ol{v}, \ol{uv}, \ol{v}, \ol{u}, \ol{u}, \ol{v}, \ol{u}
  \ea
\end{exa}

\section{Conclusion and extensions}
It is apparent that
a complete classification of three-element constructions must be more complicated than that
for two elements, as Example \ref{bsst} shows that the condition of Theorem \ref{sst}
alone is too weak. In spite of this counter-example, the lemmas included in Section \ref{secthree}
provide a promising start towards classifying multisets composed of $\ol{u}, \ol{v}, \ol{uv}$.
As in the case of two-element constructions, a complete classification of this form of multiset
would provide a stable foundation for building a classification of all three-element constructions
using automorphisms. Below we provide a conjecture on the existence of such a classification.

\begin{conj}\label{threeclass}
  In the dihedral group $D_{2m}$, multisets of the form
  $$M = \setof{a \cdot \ol{u},\ b \cdot \ol{v},\ c \cdot \ol{uv}}$$
  are realizable for exactly the values
  $$\setdef{a, b, c}{b, c \leq m,\ a, b, c \geq 1,\ a + b + c = 2m - 1}$$
\end{conj}

This conjecture implies that \textit{all admissible} multisets of the given form
are realizable. Indeed, a proof that multisets outside of the range of Conjecture \ref{threeclass}
are not realizable is easily supplied below.

\begin{lem}\label{conjnegate}
  In the dihedral group $D_{2m}$, multisets of the form
  $$M = \setof{a \cdot \ol{u},\ b \cdot \ol{v},\ c \cdot \ol{uv}}$$
  are not realizable for the values
  $$\setdef{a, b, c}{c > m \text{ or } b > m}$$
\end{lem}
\begin{proof}
  Corollary of Theorem \ref{sst}. The values are explicitly proven
  to be impossible by the proof of Corollary \ref{leqm}.
\end{proof}

The lemmas provided in Section \ref{secthree} cover almost all of Conjecture \ref{threeclass},
but there are still some cases missing. The ``missing'' realizations are supplied below
as additional conjectures, accompanied by examples indicating their possible
realizability.

\begin{conj}\label{amodd}
In the dihedral group $D_{2m}$, multisets of the form
$$M = \setof{a \cdot \ol{u},\ b \cdot \ol{v},\ c \cdot \ol{uv}}$$
are realizable for the values
$$\setdef{a, b, c}{a = m,\ a + b + c = 2m - 1}$$
when $m$ and/or $c$ are odd.
\end{conj}

\begin{exa}\label{oddexa}
  In the dihedral group $D_{10}$, the multiset
  $$M = \setof{ 5 \cdot \ol{u},\ 1 \cdot \ol{v},\ 3 \cdot \ol{uv}}$$
  is realized by the sequence
  \ba
  &Sequence : e, v, uv, u^2v, u, u^2, u^3v, u^4v, u^3, u^4\\
  &Labels : \ol{v}, \ol{u}, \ol{u}, \ol{uv}, \ol{u}, \ol{uv}, \ol{u}, \ol{uv}, \ol{u}
  \ea
\end{exa}

Note that the following conjecture overlaps slightly with Lemma \ref{morev} for the
sake of making a succinct statement.

\begin{conj}\label{vuvdiff}
In the dihedral group $D_{2m}$, multisets of the form
$$M = \setof{a \cdot \ol{u},\ b \cdot \ol{v},\ c \cdot \ol{uv}}$$
are realizable for the values
$$\setdef{a, b, c}{a \neq m,\ b,c < m,\ |b - c| \geq 2,\ a + b + c = 2m - 1}$$
\end{conj}
\begin{exa}
  In the dihedral group $D_{10}$, the multiset
  $$M = \setof{ 4 \cdot \ol{u},\ 4 \cdot \ol{v},\ 1 \cdot \ol{uv}}$$
  is realized by the sequence
  \ba
  &Sequence : e, v, uv, u, u^2v, u^2, u^3, u^4, u^4v, u^3\\
  &Labels : \ol{v}, \ol{u}, \ol{v}, \ol{uv}, \ol{v}, \ol{u}, \ol{u}, \ol{v}, \ol{u}
  \ea
\end{exa}

As demonstrated, the multiset forms supplied in Conjectures \ref{amodd} and \ref{vuvdiff}
are by no means unrealizable. These are merely the admissible forms of $\ol{u},\ol{v},\ol{uv}$
multisets that could not be covered by the realizations supplied in Section \ref{secthree}. The
examples provided above were obtained by first creating $c$ pairs of the form $\setof{u^n, u^{n+1}v}$,
then traversing the remaining elements using differences of $\ol{u}$ and $\ol{v}$. This technique most
likely generalizes to all cases of Conjecture \ref{threeclass}, and could potentially be used to compose
a single proof for all the multisets it describes.

The logical next step is determining the truth of Conjecture \ref{threeclass}, and from that point
proceeding with automorphisms on a complete classification of multisets composed of
$\ol{u}, \ol{v}, \ol{uv}$. If Conjecture \ref{threeclass} is correct, and all such admissible
multisets are realizable, then the difference between admissibility and realizability proven by
Example \ref{bsst} takes place only in three-element constructions with other elements besides
$\ol{u}, \ol{v}, \ol{uv}$. Further extensions will require an investigation of exactly when this difference
between admissibility and realizability occurs.
\pagebreak

\addcontentsline{toc}{section}{References}
\begin{thebibliography}{6}

\bibitem{BHR}
  P. Horak, A. Rosa,
  On a problem of Marco Buratti,
  \textit{Electronic J. Combin.} \textbf{16} (2009), $\sharp$R20.

\bibitem{NewResult}
  A. Pasotti, M. A. Pellegrini,
  A new result on the problem of Buratti, Horak and Rosa,
  \textit{Discrete Math.}, \textbf{319} (2014), 1-14. 

\bibitem{NewMethods}
  M. A. Ollis, A. Pasotti, M. A. Pellegrini, J. R. Schmitt,
  New methods to attack the Buratti-Horak-Rosa conjecture,
  \textit{submitted} (2020).

\bibitem{123}
  S. Capparelli, A. Del Fra,
  Hamiltonian paths in the complete graph with edge-lengths 1,2,3,
  \textit{Electron. J. Combin.} \textbf{17} (2010), $\sharp$R44.

\bibitem{OnBHR}
  A. Pasotti, M. A. Pellegrini,
  On the Buratti-Horak-Rosa Conjecture about Hamiltonian paths in complete graphs,
  \textit{Electron. J. Combin.} \textbf{21} (2014), $\sharp$P2.30.

\bibitem{SandS}
  B. Seamone, B. Stevens,
  Spanning trees with specified differences in Cayley graphs,
  \textit{Discrete Math.}, \textbf{312} (2012), 2561-2565.

\end{thebibliography}

\end{document}
