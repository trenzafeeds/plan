

\documentclass[12pt]{article}


\pagestyle{plain}

%packages
\usepackage{latexsym}
\usepackage{amsfonts}
\usepackage{amsmath}
\usepackage{amsthm}
\usepackage{amssymb}
\usepackage{fullpage}
\usepackage{changepage}
\usepackage{listings}
\usepackage{graphicx}
\graphicspath{ {../images/} }

\usepackage{dirtree}

\usepackage{abstract}
\renewcommand{\abstractname}{}
\renewcommand{\absnamepos}{empty}

%environments
\newtheorem{thm}{Theorem}
\newtheorem{exa}{Example}
\newtheorem{lem}{Lemma}
\newtheorem{cor}{Corollary}
\newtheorem{prop}{Proposition}
\newtheorem{conj}{Conjecture}
\newtheorem{defi}{Definition}

%-- notation -- Check these out!!
\newcommand{\noqed}{\begin{flushright}$\Box$\end{flushright}}
\newcommand{\ba}{\[\begin{aligned}}
\newcommand{\ea}{\end{aligned}\]}
%numbers
\newcommand{\mc}[1]{\mathcal{#1}}
\newcommand{\N}{\mathbb{N}}
\newcommand{\Z}{\mathbb{Z}}
\newcommand{\R}{\mathbb{R}}
\newcommand{\C}{\mathbb{C}}
\newcommand{\Q}{\mathbb{Q}}
%sets
\newcommand{\setof}[1]{\left\{#1\right\}}
\newcommand{\setdef}[2]{\left\{#1\mid#2\right\}}
%matrix
\newcommand{\pmat}[1]{\begin{pmatrix}#1\end{pmatrix}}
%groups
\newcommand{\gen}[1]{\langle#1\rangle} %generator
%topology
\newcommand{\Lk}{\text{Lk}}
\newcommand{\St}{\text{St}}
\newcommand{\dl}{\text{dl}}
\newcommand{\Bary}{\text{Bary }}
\newcommand{\skel}{\text{skel }}
\newcommand{\Baryn}[1]{\text{Bary}^{#1}\ }
\newcommand{\skeln}[1]{\text{skel}^{#1}\ }

\newcommand\Bstrut{\rule[-0.4cm]{0pt}{0pt}}
\newcommand\Tstrut{\rule{0pt}{0.4cm}}

%spacing
%\setlength{\parindent}{0in}
%\setlength{\parskip}{3mm}


\begin{document}


\begin{titlepage}
  \centering
      {\scshape\LARGE Combinatorics and Distributed Computing \par}
      {\scshape\large A Marlboro Plan of Concentration \par}
      \vspace{1cm}
      \includegraphics[scale=.85]{marlboro_logo.jpg}\par\vspace{1cm}
      {\scshape\LARGE Kat Cannon-MacMartin \par}
      \vspace{1cm}       
      {\scshape\Large Mathematics \& Computer Science\par}
      \vspace{.25cm}
      {\scshape\Large April of 2021 \par}
      \vspace{.1cm}

      \begin{flushleft}
        $$\begin{array}{lclcl}

            \text{{\scshape Plan Sponsor }} & 
            \text{-} & 
            \text{{\scshape Matt Ollis }} & 
            \text{-} & 
            \text{{\scshape Professor of Mathematics }} \\

            & & & & \hspace{30mm} \text{{\scshape at Marlboro College }}  \\

            \text{{\scshape Plan Sponsor }} &
            \text{-} &
            \text{{\scshape Jim Mahoney }} &
            \text{-}  &
            \text{{\scshape Professor of Computer Science }}\\

            & & & & \hspace{30mm} 

            \text{{\scshape at Marlboro College }}  \\
          
          
            \text{{\scshape Outside Evaluator }} & 
            \text{-} & \text{{\scshape Onur A\u{g}{\i}rseven}}\\
      \end{array}$$
      \end{flushleft}
      
      \section*{Abstract}
      \begin{abstract}
        {\normalsize
        This Plan of Concentration is composed of research in combinatorics,
        a survey of methods in distributed computing, and general tests covering mathematics and
        computer science. The combinatorics research paper investigates multisets of edge labels
        that form Hamiltonian paths through graphs labeled with elements of the dihedral group.
        This research expands on a question proposed by Buratti, Horak, and Rosa about cyclic groups.
        The distributed computing survey exemplifies common theoretical models for distributed algorithms
        through code examples, accompanied by a descriptive paper. Finally, a test in mathematics and a test
        in computer science demonstrate mastery of more general knowledge acquired over four years of study.
        }
      \end{abstract}

      \bigskip
      \bigskip
      
      \begin{center}
        \section*{Components}
        {\large
        \noindent\begin{tabular}{l||l}
        Component & \%\Tstrut \\
        \hline
        \hline
        Hamiltonian paths through the complete graph of the dihedral group\hspace{0.3cm}  & 50\\
        \hspace{1cm} \textit{Research paper in graph theory and combinatorics.}\Bstrut\\
        Three flavors of Paxos & 30 \\
        \hspace{1cm} \textit{Three implementations of the Paxos algorithm demonstrated}\\
        \hspace{1cm} \textit{in C, with accompanying paper.}\Bstrut\\
        Exam in Mathematics  & 10\Bstrut \\
        Exam in Computer Science  & 10

        \end{tabular}
        }
      \end{center}
\end{titlepage}

\end{document}
